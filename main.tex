\documentclass{article}
\usepackage[utf8]{inputenc}
\usepackage{amsmath,amssymb}
\usepackage{graphicx,wrapfig}
\usepackage[
    colorlinks=true,  % Makes the links colored instead of boxed
    linkcolor=blue,   % Color for internal links (e.g., table of contents)
    citecolor=green,  % Color for citations
    urlcolor=cyan,    % Color for URLs/external links
]{hyperref}
\usepackage{physics}
\usepackage{geometry} % For setting margins
\geometry{a4paper, margin=1in} % 1-inch margins for a professional look
\usepackage{setspace} % For line spacing
\setstretch{1.2} % Set line spacing to 1.2

\begin{document}
%Title Page and certificate and acknowlwdgements
\newgeometry{top=5cm,bottom=2.5cm,left=4cm,right=4cm}
\begin{titlepage}

\begin{center}



\textup{\Large {\bf Spherical Model of a Ferromagnet}}\\[0.3in]

% Title
\textbf {Phase Transitions and Critical Phenomena (P455): Term Paper}\\[0.7in]


       

% Submitted by
\normalsize Submitted by \\
\textbf{Ashlin V Thomas (Roll No: 2211030)}\\
\textbf{Roshan Mathew Philip (Roll No: 2211192)}\\
\textbf{Satwik Yadav (Roll No: )}\\
\textbf{Yash Chauhan (Roll No: )}\\
\normalsize
\vspace{1 cm}

\includegraphics[width=0.25 \textwidth]{Images/NISER.png}\\[0.1in]
\Large{School of Physical Sciences}\\
\normalsize
\textsc{National Institute of Science Education and Research},\\
Tehsildar Office, Khurda\\
Pipli, Near, Jatni, Odisha 752050\\

\vspace{.4in}
Submitted to \\
\large\textbf{Dr. Sumedha}\\
\normalsize
Associate Professor

% Bottom of the page
\Large{School of Physical Sciences}\\
\normalsize
\textsc{National Institute of Science Education and Research},\\
Tehsildar Office, Khurda\\
Pipli, Near, Jatni, Odisha 752050\\
\end{center}

\end{titlepage}
\restoregeometry




%Abstract
\newgeometry{top=6.5cm,bottom=2.5cm,left=5cm,right=5cm}
\cleardoublepage
\begin{center}
    \Large{\textbf{Acknowledgements}}
\end{center}

\vspace{0.2in}

We would like to express our profound gratitude to our course instructor Dr.Sumedha and the School of Physical Sciences for providing us with an opportunity to undertake this project. Support and guidance provided by the instructor during the coursework and project was critical towards the successful completion of this project.

\newpage
\restoregeometry


\newgeometry{top=6.5cm,bottom=2.5cm,left=3.5cm,right=3.5cm}
\begin{abstract}

\end{abstract}
\restoregeometry



%Content Table Page
\newpage
\tableofcontents




\newpage
\pagenumbering{arabic}
\section{Introduction}

In the study of phase transitions, especially how a material like a ferromagnet spontaneously develops magnetization below its critical temperature, has always been a challenging problem in statistical mechanics. Numerous efforts have been made to create models that can accurately describe this behavior, but many of these models are too complex to solve exactly, especially in three dimensions. A notable example is the 3 dimensional Ising model, which remains unsolved to this day.
That is where the Spherical Model of a Ferromagnet\cite{Foundational}, proposed by T. H. Berlin and M. Kac in 1952, comes into the play. This model simplifies the problem by relaxing the strict constraints on individual spins, allowing for an exact solution in arbitrary dimensions, allowing us to explore critical phenomena and phase transitions in a three dimensional setting.

In the classic Ising model, each spin can only take on discrete values of $+1$ or $-1$. Berlin and Kac relaxed this constraint by allowing the spins to take continuous values by replacing it with a global constraint on the sum of the squares of the spins. This made the model mathematically tractable in arbitrary dimensions, and exhibited a phase transition in all dimensions greater than two. 
Apart from the exact solution, the model proved to be a valuable tool for understanding Bose-Einstein condensation\cite{BECanalogy} and also in the study of critical phenomena in systems of finite thickness\cite{finitethickness}. 

\section{The Gaussian Model}
The Spherical Model was introduced as a modification of the Gaussian model by Berlin and Kac. The Gaussian model is a continuous version of the Ising model where the spins can take any real value, and the Hamiltonian is given by:
\begin{equation}
    H = -J \sum_{\langle i,j \rangle} \epsilon_i \epsilon_j
\end{equation}
where $J$ is the interaction strength, and the sum is over nearest neighbor pairs. Each spin $\epsilon_i$ is drawn from a Gaussian distribution with zero mean and unit variance. That is, the probability of finding the $i^{th}$ spin with between $\epsilon_i$ and $\epsilon_i + d\epsilon_i$ is given by:
\begin{equation}
    P(\epsilon_i) d\epsilon_i = \frac{1}{\sqrt{2\pi}} e^{-\frac{\epsilon_i^2}{2}} d\epsilon_i
\end{equation}
This yields the partition function as:
\begin{align}
    Z &=  \frac{1}{(2\pi)^{N/2}} \int_{-\infty}^{\infty} \ldots \int_{-\infty}^{\infty} \left( \prod_{i=1}^{N} d\epsilon_i \right) exp \left( -\frac{1}{2} \sum_{i=1}^{N} \epsilon_i^2 + K \sum_{\langle i,j \rangle} \epsilon_i \epsilon_j \right) \\
     & = exp \left( -\frac{1}{2} \sum_{p=1}^{N} \ln(1 - 2 K \lambda_p) \right)
\end{align}
where $K = \beta J$, and $\lambda_p$ for different lattices are given in the table below:
\begin{table}[h!]
    \centering
    \begin{tabular}{|c|c|}
        \hline
        Lattice & $\lambda_p$ \\
        \hline \hline
        1-dimensional chain & $2 cos\left(\frac{2 \pi (p-1)}{N} \right)$ \\
        \hline
        2-dimensional square lattice & 2 $cos\left(\frac{2 \pi (p-1)}{N} \right) + 2 cos\left(\frac{2 \pi n_1 (p-1) }{N} \right)$ \\
        \hline
        3-dimensional cubic lattice & $2 cos\left(\frac{2 \pi (p-1)}{N} \right) + 2 cos\left(\frac{2 \pi n_1 (p-1) }{N} \right) + 2 cos\left(\frac{2 \pi n_1 n_2 (p-1) }{N} \right)$ \\
        \hline
    \end{tabular}
    \caption{Values of $\lambda_p$ for different lattices}
    \label{tab:lambda}
\end{table}


Here, $n_1$ and $n_2$ are the number of sites along the two dimensions of the 2D lattice, and $N = n_1 n_2$ is the total number of sites. Similarly, for the 3D lattice, $n_1$, $n_2$, and $n_3$ are the number of sites along the three dimensions, and $N = n_1 n_2 n_3$ is the total number of sites.

The free energy per spin in the thermodynamic limit can be found using the partition function as:
\begin{equation}
    \psi = \frac{k_B T}{2} \lim_{N \rightarrow \infty} \frac{1}{N} \sum_{p=1}^{N} \ln(1 - 2 K \lambda_p)
\end{equation}
On substituting the values of $\lambda_p$ from Table \ref{tab:lambda}, we find that the free energy per spin for the 1D, 2D, and 3D lattices are in agreement with the form of the free energy obtained from the Ising model. 
Despite this, the free energy becomes complex for $K > K_c = \frac{1}{4 n}$, where $n$ is the dimensionality of the lattice. This indicates that the Gaussian model is not well defined below a critical temperature. So,
we introduce the spherical model, which retains all important features of the Gaussian model and is valid for all temperatures.

\section
A spin configuration $\mathbf{\epsilon} = (\epsilon_1, \epsilon_2, \ldots, \epsilon_N) \in \mathbb{R}^N$ is an allowed configuration in the Ising model if $\epsilon_i = \pm 1$ for all $1 \leq i \leq N$. That is, $\mathbf{\epsilon}$ lies on the vertices of an $N$-dimensional hypercube of side length 2 units centered at the origin. 
Consider a sphere of radius $\sqrt{N}$ centered at the origin in $\mathbb{R}^N$:
\begin{equation}
S_{N-1}(\sqrt{N}) = \{ \mathbf{\epsilon} \in \mathbb{R}^N : \sum_{i=1}^{N} \epsilon_i^2 = N \} 
\end{equation}
Clearly, all allowed configurations of the Ising model lie on this sphere. The spherical model allows every point on $S_{N-1}(\sqrt{N})$ to be an allowed configuration. \\

The spherical constraint can be modelled in the partition function as follows:
\begin{align}
    Z = & \frac{\Gamma(N/2)}{2(\pi)^{N/2} N^{(N-1)/2}} \int_{-\infty}^{\infty} \ldots \int_{-\infty}^{\infty} \left( \prod_{i=1}^{N} d\epsilon_i \right) exp \left( K \sum_{\langle i,j \rangle} \epsilon_i \epsilon_j \right) \delta \left( N - \sum_{i=1}^{N} \epsilon_i^2 \right) 
\end{align}












\section{Spherical Model in a Homogeneous Magnetic Field}
The model Hamiltonian can be modified to include the effect of an external homogeneous magnetic field $H$ as follows:
\begin{equation}
    H = -J \sum_{\langle i,j \rangle} \epsilon_i \epsilon_j - H \mu_0 \sum_{i=1}^{N} \epsilon_i = -J \sum_{\langle i,j \rangle} \epsilon_i \epsilon_j - H \mu_0 \sqrt{N} y_1
\end{equation}
where $\mu_0$ is the magnetic moment associated with a single spin, and $y_1 = \frac{1}{\sqrt{N}} \sum_{i=1}^{N} \epsilon_i$. \\
In the presence of the magnetic field, the partition function becomes:
\begin{align}
    Z_H = & \frac{1}{A_N} \frac{1}{2 \pi i} \int_{-i \infty}^{i \infty} ds \ e^{s N} \int_{-\infty}^{\infty} \ldots \int_{-\infty}^{\infty} \left( \prod_{i=1}^{N} dy_i \right) exp \left( -\sum_{j=1}^{N} y_j^2 (s - K \lambda_j) + 2 \sqrt{N} M y_1 \right)
\end{align}
where $M = \frac{H \mu_0}{2 k_B T}$. \\
The $y_1$ integral can be evaluated by completing the square in the exponent, yielding a factor of $exp \left( \frac{N M^2}{s - K \lambda_1} \right)$ in the integrand. The remaining integrals can be evaluated as before, as the only change is in the prefactor of $s$ in the exponent. 
Following similar strategy, one can find the free energy per spin in the thermodynamic limit as:
\begin{equation}
    \psi_H = \frac{k_B T}{2} \left( 1 + ln(4 K) + 4 K z_s - f_n(z_s) + \frac{M^2}{K(z_s- \lambda_1/2)} \right)
\end{equation}
where $z_s = z_s(K,H)$ is the solution of the saddle point equation:
\begin{equation}
    4K = \left( \dv{f_n}{z} \right)_{z=z_s} - \frac{M^2}{K (z_s- \lambda_1/2)^2}
\end{equation}

\bibliography{ref.bib} 
\bibliographystyle{ieeetr}
\nocite{*}

\end{document}